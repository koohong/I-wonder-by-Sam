% Options for packages loaded elsewhere
\PassOptionsToPackage{unicode}{hyperref}
\PassOptionsToPackage{hyphens}{url}
%
\documentclass[
]{article}
\usepackage{amsmath,amssymb}
\usepackage{iftex}
\ifPDFTeX
  \usepackage[T1]{fontenc}
  \usepackage[utf8]{inputenc}
  \usepackage{textcomp} % provide euro and other symbols
\else % if luatex or xetex
  \usepackage{unicode-math} % this also loads fontspec
  \defaultfontfeatures{Scale=MatchLowercase}
  \defaultfontfeatures[\rmfamily]{Ligatures=TeX,Scale=1}
\fi
\usepackage{lmodern}
\ifPDFTeX\else
  % xetex/luatex font selection
\fi
% Use upquote if available, for straight quotes in verbatim environments
\IfFileExists{upquote.sty}{\usepackage{upquote}}{}
\IfFileExists{microtype.sty}{% use microtype if available
  \usepackage[]{microtype}
  \UseMicrotypeSet[protrusion]{basicmath} % disable protrusion for tt fonts
}{}
\makeatletter
\@ifundefined{KOMAClassName}{% if non-KOMA class
  \IfFileExists{parskip.sty}{%
    \usepackage{parskip}
  }{% else
    \setlength{\parindent}{0pt}
    \setlength{\parskip}{6pt plus 2pt minus 1pt}}
}{% if KOMA class
  \KOMAoptions{parskip=half}}
\makeatother
\usepackage{xcolor}
\usepackage[margin=1in]{geometry}
\usepackage{color}
\usepackage{fancyvrb}
\newcommand{\VerbBar}{|}
\newcommand{\VERB}{\Verb[commandchars=\\\{\}]}
\DefineVerbatimEnvironment{Highlighting}{Verbatim}{commandchars=\\\{\}}
% Add ',fontsize=\small' for more characters per line
\usepackage{framed}
\definecolor{shadecolor}{RGB}{248,248,248}
\newenvironment{Shaded}{\begin{snugshade}}{\end{snugshade}}
\newcommand{\AlertTok}[1]{\textcolor[rgb]{0.94,0.16,0.16}{#1}}
\newcommand{\AnnotationTok}[1]{\textcolor[rgb]{0.56,0.35,0.01}{\textbf{\textit{#1}}}}
\newcommand{\AttributeTok}[1]{\textcolor[rgb]{0.13,0.29,0.53}{#1}}
\newcommand{\BaseNTok}[1]{\textcolor[rgb]{0.00,0.00,0.81}{#1}}
\newcommand{\BuiltInTok}[1]{#1}
\newcommand{\CharTok}[1]{\textcolor[rgb]{0.31,0.60,0.02}{#1}}
\newcommand{\CommentTok}[1]{\textcolor[rgb]{0.56,0.35,0.01}{\textit{#1}}}
\newcommand{\CommentVarTok}[1]{\textcolor[rgb]{0.56,0.35,0.01}{\textbf{\textit{#1}}}}
\newcommand{\ConstantTok}[1]{\textcolor[rgb]{0.56,0.35,0.01}{#1}}
\newcommand{\ControlFlowTok}[1]{\textcolor[rgb]{0.13,0.29,0.53}{\textbf{#1}}}
\newcommand{\DataTypeTok}[1]{\textcolor[rgb]{0.13,0.29,0.53}{#1}}
\newcommand{\DecValTok}[1]{\textcolor[rgb]{0.00,0.00,0.81}{#1}}
\newcommand{\DocumentationTok}[1]{\textcolor[rgb]{0.56,0.35,0.01}{\textbf{\textit{#1}}}}
\newcommand{\ErrorTok}[1]{\textcolor[rgb]{0.64,0.00,0.00}{\textbf{#1}}}
\newcommand{\ExtensionTok}[1]{#1}
\newcommand{\FloatTok}[1]{\textcolor[rgb]{0.00,0.00,0.81}{#1}}
\newcommand{\FunctionTok}[1]{\textcolor[rgb]{0.13,0.29,0.53}{\textbf{#1}}}
\newcommand{\ImportTok}[1]{#1}
\newcommand{\InformationTok}[1]{\textcolor[rgb]{0.56,0.35,0.01}{\textbf{\textit{#1}}}}
\newcommand{\KeywordTok}[1]{\textcolor[rgb]{0.13,0.29,0.53}{\textbf{#1}}}
\newcommand{\NormalTok}[1]{#1}
\newcommand{\OperatorTok}[1]{\textcolor[rgb]{0.81,0.36,0.00}{\textbf{#1}}}
\newcommand{\OtherTok}[1]{\textcolor[rgb]{0.56,0.35,0.01}{#1}}
\newcommand{\PreprocessorTok}[1]{\textcolor[rgb]{0.56,0.35,0.01}{\textit{#1}}}
\newcommand{\RegionMarkerTok}[1]{#1}
\newcommand{\SpecialCharTok}[1]{\textcolor[rgb]{0.81,0.36,0.00}{\textbf{#1}}}
\newcommand{\SpecialStringTok}[1]{\textcolor[rgb]{0.31,0.60,0.02}{#1}}
\newcommand{\StringTok}[1]{\textcolor[rgb]{0.31,0.60,0.02}{#1}}
\newcommand{\VariableTok}[1]{\textcolor[rgb]{0.00,0.00,0.00}{#1}}
\newcommand{\VerbatimStringTok}[1]{\textcolor[rgb]{0.31,0.60,0.02}{#1}}
\newcommand{\WarningTok}[1]{\textcolor[rgb]{0.56,0.35,0.01}{\textbf{\textit{#1}}}}
\usepackage{graphicx}
\makeatletter
\def\maxwidth{\ifdim\Gin@nat@width>\linewidth\linewidth\else\Gin@nat@width\fi}
\def\maxheight{\ifdim\Gin@nat@height>\textheight\textheight\else\Gin@nat@height\fi}
\makeatother
% Scale images if necessary, so that they will not overflow the page
% margins by default, and it is still possible to overwrite the defaults
% using explicit options in \includegraphics[width, height, ...]{}
\setkeys{Gin}{width=\maxwidth,height=\maxheight,keepaspectratio}
% Set default figure placement to htbp
\makeatletter
\def\fps@figure{htbp}
\makeatother
\setlength{\emergencystretch}{3em} % prevent overfull lines
\providecommand{\tightlist}{%
  \setlength{\itemsep}{0pt}\setlength{\parskip}{0pt}}
\setcounter{secnumdepth}{-\maxdimen} % remove section numbering
\ifLuaTeX
  \usepackage{selnolig}  % disable illegal ligatures
\fi
\IfFileExists{bookmark.sty}{\usepackage{bookmark}}{\usepackage{hyperref}}
\IfFileExists{xurl.sty}{\usepackage{xurl}}{} % add URL line breaks if available
\urlstyle{same}
\hypersetup{
  pdftitle={Linear Algebrea: wk5 projection},
  pdfauthor={Bill Chung},
  hidelinks,
  pdfcreator={LaTeX via pandoc}}

\title{Linear Algebrea: wk5 projection}
\author{Bill Chung}
\date{3월 08, 2024}

\begin{document}
\maketitle

\begin{Shaded}
\begin{Highlighting}[]
\FunctionTok{library}\NormalTok{(far)}
\FunctionTok{library}\NormalTok{(MASS)}
\FunctionTok{library}\NormalTok{(pracma)}
\FunctionTok{library}\NormalTok{(expm)}
\end{Highlighting}
\end{Shaded}

\hypertarget{example}{%
\section{Example}\label{example}}

\begin{Shaded}
\begin{Highlighting}[]
\NormalTok{v1 }\OtherTok{\textless{}{-}} \FunctionTok{c}\NormalTok{(}\DecValTok{1}\NormalTok{,}\DecValTok{2}\NormalTok{,}\DecValTok{3}\NormalTok{)}
\NormalTok{v2 }\OtherTok{\textless{}{-}} \FunctionTok{c}\NormalTok{(}\DecValTok{1}\NormalTok{,}\DecValTok{0}\NormalTok{,}\SpecialCharTok{{-}}\DecValTok{1}\NormalTok{)}
\NormalTok{v3 }\OtherTok{\textless{}{-}} \FunctionTok{c}\NormalTok{(}\DecValTok{1}\NormalTok{,}\DecValTok{0}\NormalTok{,}\DecValTok{1}\NormalTok{)}

\NormalTok{A }\OtherTok{\textless{}{-}} \FunctionTok{cbind}\NormalTok{(v1,v2,v3)}
\FunctionTok{print}\NormalTok{(A)}
\end{Highlighting}
\end{Shaded}

\begin{verbatim}
##      v1 v2 v3
## [1,]  1  1  1
## [2,]  2  0  0
## [3,]  3 -1  1
\end{verbatim}

\begin{Shaded}
\begin{Highlighting}[]
\NormalTok{b }\OtherTok{\textless{}{-}} \FunctionTok{c}\NormalTok{(}\DecValTok{3}\NormalTok{,}\DecValTok{2}\NormalTok{,}\DecValTok{5}\NormalTok{)}

\NormalTok{Ab }\OtherTok{\textless{}{-}} \FunctionTok{cbind}\NormalTok{(A,b)}
\FunctionTok{print}\NormalTok{(Ab)  }
\end{Highlighting}
\end{Shaded}

\begin{verbatim}
##      v1 v2 v3 b
## [1,]  1  1  1 3
## [2,]  2  0  0 2
## [3,]  3 -1  1 5
\end{verbatim}

\begin{Shaded}
\begin{Highlighting}[]
\CommentTok{\#step 1}
\FunctionTok{Rank}\NormalTok{(A)}
\end{Highlighting}
\end{Shaded}

\begin{verbatim}
## [1] 3
\end{verbatim}

\begin{Shaded}
\begin{Highlighting}[]
\FunctionTok{Rank}\NormalTok{(Ab)}
\end{Highlighting}
\end{Shaded}

\begin{verbatim}
## [1] 3
\end{verbatim}

\begin{Shaded}
\begin{Highlighting}[]
\CommentTok{\#step 2}
\NormalTok{x }\OtherTok{\textless{}{-}} \FunctionTok{inv}\NormalTok{(A)}\SpecialCharTok{\%*\%}\NormalTok{b}

\NormalTok{A}\SpecialCharTok{\%*\%}\NormalTok{x}
\end{Highlighting}
\end{Shaded}

\begin{verbatim}
##      [,1]
## [1,]    3
## [2,]    2
## [3,]    5
\end{verbatim}

\begin{Shaded}
\begin{Highlighting}[]
\NormalTok{v1 }\OtherTok{\textless{}{-}} \FunctionTok{c}\NormalTok{(}\DecValTok{1}\NormalTok{,}\DecValTok{2}\NormalTok{,}\DecValTok{5}\NormalTok{,}\DecValTok{7}\NormalTok{,}\DecValTok{6}\NormalTok{)}
\NormalTok{v2 }\OtherTok{\textless{}{-}} \FunctionTok{c}\NormalTok{(}\DecValTok{10}\NormalTok{,}\DecValTok{5}\NormalTok{,}\DecValTok{0}\NormalTok{,}\DecValTok{2}\NormalTok{,}\DecValTok{3}\NormalTok{)}
\NormalTok{v3 }\OtherTok{\textless{}{-}} \FunctionTok{c}\NormalTok{(}\SpecialCharTok{{-}}\DecValTok{1}\NormalTok{,}\FloatTok{0.2}\NormalTok{,}\DecValTok{100}\NormalTok{,}\SpecialCharTok{{-}}\DecValTok{4}\NormalTok{,}\DecValTok{7}\NormalTok{)}

\NormalTok{A }\OtherTok{\textless{}{-}} \FunctionTok{cbind}\NormalTok{(v1,v2,v3)}
\FunctionTok{print}\NormalTok{(A)}
\end{Highlighting}
\end{Shaded}

\begin{verbatim}
##      v1 v2    v3
## [1,]  1 10  -1.0
## [2,]  2  5   0.2
## [3,]  5  0 100.0
## [4,]  7  2  -4.0
## [5,]  6  3   7.0
\end{verbatim}

\begin{Shaded}
\begin{Highlighting}[]
\NormalTok{b }\OtherTok{\textless{}{-}} \FunctionTok{c}\NormalTok{(}\DecValTok{3}\NormalTok{,}\DecValTok{4}\NormalTok{,}\SpecialCharTok{{-}}\DecValTok{300}\NormalTok{,}\FloatTok{0.5}\NormalTok{,}\DecValTok{10}\NormalTok{)}

\NormalTok{Ab }\OtherTok{\textless{}{-}} \FunctionTok{cbind}\NormalTok{(A,b)}

\CommentTok{\#step 1}
\FunctionTok{Rank}\NormalTok{(A)}
\end{Highlighting}
\end{Shaded}

\begin{verbatim}
## [1] 3
\end{verbatim}

\begin{Shaded}
\begin{Highlighting}[]
\FunctionTok{Rank}\NormalTok{(Ab)}
\end{Highlighting}
\end{Shaded}

\begin{verbatim}
## [1] 4
\end{verbatim}

\begin{Shaded}
\begin{Highlighting}[]
\CommentTok{\#find something in my column space A that is close to b}
\NormalTok{G }\OtherTok{\textless{}{-}} \FunctionTok{t}\NormalTok{(A)}\SpecialCharTok{\%*\%}\NormalTok{A }\CommentTok{\#gram matrix}
\NormalTok{x }\OtherTok{\textless{}{-}} \FunctionTok{inv}\NormalTok{(G)}\SpecialCharTok{\%*\%}\FunctionTok{t}\NormalTok{(A)}\SpecialCharTok{\%*\%}\NormalTok{b}

\NormalTok{b\_hat }\OtherTok{\textless{}{-}}\NormalTok{ A}\SpecialCharTok{\%*\%}\NormalTok{x}


\NormalTok{P }\OtherTok{\textless{}{-}}\NormalTok{ A}\SpecialCharTok{\%*\%}\FunctionTok{inv}\NormalTok{(G)}\SpecialCharTok{\%*\%}\FunctionTok{t}\NormalTok{(A)}
\NormalTok{b\_hat }\OtherTok{\textless{}{-}}\NormalTok{P}\SpecialCharTok{\%*\%}\NormalTok{b}

\NormalTok{N }\OtherTok{\textless{}{-}} \FunctionTok{diag}\NormalTok{(}\DecValTok{5}\NormalTok{) }\SpecialCharTok{{-}}\NormalTok{ P}

\NormalTok{residual }\OtherTok{\textless{}{-}}\NormalTok{ N}\SpecialCharTok{\%*\%}\NormalTok{b}
\NormalTok{b}
\end{Highlighting}
\end{Shaded}

\begin{verbatim}
## [1]    3.0    4.0 -300.0    0.5   10.0
\end{verbatim}

\begin{Shaded}
\begin{Highlighting}[]
\NormalTok{b\_hat }\SpecialCharTok{+}\NormalTok{ residual}
\end{Highlighting}
\end{Shaded}

\begin{verbatim}
##        [,1]
## [1,]    3.0
## [2,]    4.0
## [3,] -300.0
## [4,]    0.5
## [5,]   10.0
\end{verbatim}

\begin{Shaded}
\begin{Highlighting}[]
\FunctionTok{round}\NormalTok{(}\FunctionTok{t}\NormalTok{(b\_hat)}\SpecialCharTok{\%*\%}\NormalTok{residual,}\DecValTok{0}\NormalTok{)}
\end{Highlighting}
\end{Shaded}

\begin{verbatim}
##      [,1]
## [1,]    0
\end{verbatim}

\begin{Shaded}
\begin{Highlighting}[]
\NormalTok{c1 }\OtherTok{\textless{}{-}} \FunctionTok{c}\NormalTok{(}\DecValTok{1}\NormalTok{,}\DecValTok{2}\NormalTok{,}\DecValTok{3}\NormalTok{)}
\NormalTok{c2 }\OtherTok{\textless{}{-}} \FunctionTok{c}\NormalTok{(}\DecValTok{1}\NormalTok{,}\DecValTok{2}\NormalTok{,}\SpecialCharTok{{-}}\DecValTok{3}\NormalTok{)}
\NormalTok{c3 }\OtherTok{\textless{}{-}} \FunctionTok{c}\NormalTok{(}\SpecialCharTok{{-}}\DecValTok{1}\NormalTok{,}\DecValTok{2}\NormalTok{,}\DecValTok{3}\NormalTok{)}
\NormalTok{c4 }\OtherTok{\textless{}{-}} \FunctionTok{c}\NormalTok{(}\DecValTok{1}\NormalTok{,}\SpecialCharTok{{-}}\FloatTok{2.1}\NormalTok{,}\FloatTok{3.01}\NormalTok{)}
\NormalTok{c5 }\OtherTok{\textless{}{-}} \FunctionTok{c}\NormalTok{(}\DecValTok{1}\NormalTok{,}\DecValTok{2}\NormalTok{,}\FloatTok{0.03}\NormalTok{)}

\NormalTok{A }\OtherTok{\textless{}{-}} \FunctionTok{cbind}\NormalTok{(c1,c2,c3,c4,c5)}

\FunctionTok{print}\NormalTok{(A)}
\end{Highlighting}
\end{Shaded}

\begin{verbatim}
##      c1 c2 c3    c4   c5
## [1,]  1  1 -1  1.00 1.00
## [2,]  2  2  2 -2.10 2.00
## [3,]  3 -3  3  3.01 0.03
\end{verbatim}

\begin{Shaded}
\begin{Highlighting}[]
\FunctionTok{Rank}\NormalTok{(A)}
\end{Highlighting}
\end{Shaded}

\begin{verbatim}
## [1] 3
\end{verbatim}

\begin{Shaded}
\begin{Highlighting}[]
\NormalTok{b }\OtherTok{\textless{}{-}} \FunctionTok{c}\NormalTok{(}\DecValTok{4}\NormalTok{,}\DecValTok{3}\NormalTok{,}\DecValTok{5}\NormalTok{)}

\CommentTok{\#step divide A into B and D}
\FunctionTok{rref}\NormalTok{(A)}
\end{Highlighting}
\end{Shaded}

\begin{verbatim}
##      c1 c2 c3        c4    c5
## [1,]  1  0  0  1.001667 0.505
## [2,]  0  1  0 -1.026667 0.495
## [3,]  0  0  1 -1.025000 0.000
\end{verbatim}

\begin{Shaded}
\begin{Highlighting}[]
\NormalTok{B }\OtherTok{\textless{}{-}}\NormalTok{ A[,}\FunctionTok{c}\NormalTok{(}\DecValTok{1}\NormalTok{,}\DecValTok{2}\NormalTok{,}\DecValTok{3}\NormalTok{)]}
\NormalTok{B}
\end{Highlighting}
\end{Shaded}

\begin{verbatim}
##      c1 c2 c3
## [1,]  1  1 -1
## [2,]  2  2  2
## [3,]  3 -3  3
\end{verbatim}

\begin{Shaded}
\begin{Highlighting}[]
\NormalTok{G }\OtherTok{\textless{}{-}} \FunctionTok{t}\NormalTok{(B)}\SpecialCharTok{\%*\%}\NormalTok{B}

\NormalTok{D }\OtherTok{\textless{}{-}}\NormalTok{ A[,}\SpecialCharTok{{-}}\FunctionTok{c}\NormalTok{(}\DecValTok{1}\NormalTok{,}\DecValTok{2}\NormalTok{,}\DecValTok{3}\NormalTok{)]}
\NormalTok{D}
\end{Highlighting}
\end{Shaded}

\begin{verbatim}
##         c4   c5
## [1,]  1.00 1.00
## [2,] -2.10 2.00
## [3,]  3.01 0.03
\end{verbatim}

\begin{Shaded}
\begin{Highlighting}[]
\NormalTok{x\_B }\OtherTok{\textless{}{-}} \FunctionTok{inv}\NormalTok{(G)}\SpecialCharTok{\%*\%}\FunctionTok{t}\NormalTok{(B)}\SpecialCharTok{\%*\%}\NormalTok{b}

\NormalTok{B}\SpecialCharTok{\%*\%}\NormalTok{x\_B}
\end{Highlighting}
\end{Shaded}

\begin{verbatim}
##      [,1]
## [1,]    4
## [2,]    3
## [3,]    5
\end{verbatim}

\begin{Shaded}
\begin{Highlighting}[]
\NormalTok{x }\OtherTok{\textless{}{-}}\FunctionTok{rbind}\NormalTok{(x\_B,}\DecValTok{0}\NormalTok{,}\DecValTok{0}\NormalTok{)}
\NormalTok{x}
\end{Highlighting}
\end{Shaded}

\begin{verbatim}
##           [,1]
## c1  2.83333333
## c2 -0.08333333
## c3 -1.25000000
##     0.00000000
##     0.00000000
\end{verbatim}

\begin{Shaded}
\begin{Highlighting}[]
\NormalTok{A}\SpecialCharTok{\%*\%}\NormalTok{x}
\end{Highlighting}
\end{Shaded}

\begin{verbatim}
##      [,1]
## [1,]    4
## [2,]    3
## [3,]    5
\end{verbatim}

\begin{Shaded}
\begin{Highlighting}[]
\NormalTok{A }\OtherTok{\textless{}{-}} \FunctionTok{matrix}\NormalTok{(}\FunctionTok{rnorm}\NormalTok{(}\DecValTok{30}\NormalTok{), }\AttributeTok{nrow=}\DecValTok{10}\NormalTok{)}
\FunctionTok{Rank}\NormalTok{(A)}
\end{Highlighting}
\end{Shaded}

\begin{verbatim}
## [1] 3
\end{verbatim}

\begin{Shaded}
\begin{Highlighting}[]
\FunctionTok{class}\NormalTok{(A)}
\end{Highlighting}
\end{Shaded}

\begin{verbatim}
## [1] "matrix" "array"
\end{verbatim}

\begin{Shaded}
\begin{Highlighting}[]
\NormalTok{A}
\end{Highlighting}
\end{Shaded}

\begin{verbatim}
##             [,1]       [,2]        [,3]
##  [1,]  0.6877389  0.6083669  0.28521448
##  [2,] -0.8963631 -1.3884269 -0.55726946
##  [3,] -0.8227823 -0.1427122 -0.26880384
##  [4,]  0.6196055  0.4837353  0.08355694
##  [5,]  0.4038117  0.6593052  1.65693257
##  [6,]  1.4352106  1.7666268  1.78048647
##  [7,]  1.0137339  0.2076354 -0.20726894
##  [8,]  1.0299175  0.7212924 -0.65528190
##  [9,]  0.1450796 -0.7658370  1.42612593
## [10,]  0.3284372  1.4986480 -1.69656179
\end{verbatim}

\begin{Shaded}
\begin{Highlighting}[]
\NormalTok{b }\OtherTok{\textless{}{-}} \FunctionTok{c}\NormalTok{(}\DecValTok{1}\NormalTok{,}\DecValTok{2}\NormalTok{,}\DecValTok{3}\NormalTok{,}\DecValTok{4}\NormalTok{,}\DecValTok{5}\NormalTok{,}\DecValTok{6}\NormalTok{,}\DecValTok{7}\NormalTok{,}\DecValTok{8}\NormalTok{,}\DecValTok{9}\NormalTok{,}\DecValTok{10}\NormalTok{)}
\NormalTok{b}
\end{Highlighting}
\end{Shaded}

\begin{verbatim}
##  [1]  1  2  3  4  5  6  7  8  9 10
\end{verbatim}

\begin{Shaded}
\begin{Highlighting}[]
\NormalTok{Ab }\OtherTok{\textless{}{-}} \FunctionTok{cbind}\NormalTok{(A,b)}
\CommentTok{\#step compare the rank}
\FunctionTok{Rank}\NormalTok{(A)}
\end{Highlighting}
\end{Shaded}

\begin{verbatim}
## [1] 3
\end{verbatim}

\begin{Shaded}
\begin{Highlighting}[]
\FunctionTok{Rank}\NormalTok{(Ab)}
\end{Highlighting}
\end{Shaded}

\begin{verbatim}
## [1] 4
\end{verbatim}

\begin{Shaded}
\begin{Highlighting}[]
\CommentTok{\#project it}
\NormalTok{G }\OtherTok{\textless{}{-}} \FunctionTok{t}\NormalTok{(A)}\SpecialCharTok{\%*\%}\NormalTok{A}

\NormalTok{x }\OtherTok{\textless{}{-}} \FunctionTok{inv}\NormalTok{(G)}\SpecialCharTok{\%*\%}\FunctionTok{t}\NormalTok{(A)}\SpecialCharTok{\%*\%}\NormalTok{b}

\NormalTok{x   }
\end{Highlighting}
\end{Shaded}

\begin{verbatim}
##            [,1]
## [1,]  4.4277275
## [2,]  0.1723709
## [3,] -0.5475753
\end{verbatim}

\begin{Shaded}
\begin{Highlighting}[]
\NormalTok{A}\SpecialCharTok{\%*\%}\NormalTok{x}
\end{Highlighting}
\end{Shaded}

\begin{verbatim}
##             [,1]
##  [1,]  2.9938090
##  [2,] -3.9030292
##  [3,] -3.5204651
##  [4,]  2.7810724
##  [5,]  0.9943181
##  [6,]  5.6842862
##  [7,]  4.6378232
##  [8,]  5.0433402
##  [9,] -0.2705466
## [10,]  2.6415489
\end{verbatim}

\begin{Shaded}
\begin{Highlighting}[]
\NormalTok{A}\SpecialCharTok{\%*\%}\NormalTok{x}
\end{Highlighting}
\end{Shaded}

\begin{verbatim}
##             [,1]
##  [1,]  2.9938090
##  [2,] -3.9030292
##  [3,] -3.5204651
##  [4,]  2.7810724
##  [5,]  0.9943181
##  [6,]  5.6842862
##  [7,]  4.6378232
##  [8,]  5.0433402
##  [9,] -0.2705466
## [10,]  2.6415489
\end{verbatim}

\hypertarget{clt}{%
\section{CLT}\label{clt}}

\hypertarget{sample-satistics}{%
\subsubsection{Sample Satistics}\label{sample-satistics}}

\begin{itemize}
\tightlist
\item
  page 96
\end{itemize}

What does the following notation mean?

\[T_{(n)} = h_{(n)}(X_1,X_2,...X_n)\] where
\(h_{(n)}:\mathbb{R}^n \rightarrow \mathbb{R}, \forall n \in \mathbb{n}\)

\hypertarget{convergence-in-probability}{%
\subsubsection{Convergence in
Probability}\label{convergence-in-probability}}

\begin{itemize}
\tightlist
\item
  see page 99
\end{itemize}

\[T_{(n)} \xrightarrow p c\]

\hypertarget{weak-law-of-large-numbers}{%
\subsubsection{Weak Law of large
numbers}\label{weak-law-of-large-numbers}}

\begin{itemize}
\tightlist
\item
  see page 100 and read first two lines of page 101
\end{itemize}

\[\bar{X}_{(n)} \xrightarrow p E[X]\]

\hypertarget{law-of-large-numbers-lln}{%
\subsubsection{Law of Large Numbers
(LLN)}\label{law-of-large-numbers-lln}}

\[\lim\limits_{n \rightarrow \infty} \sum_{i=1}^{n}\frac{X_i}{n}=E[X]\]

\hypertarget{clt-1}{%
\subsubsection{CLT}\label{clt-1}}

\begin{itemize}
\tightlist
\item
  See page 109
\end{itemize}

Suppose \{\(X_1, X_2,..X_n\)\} is a sequence of iid rv with
\(\text{E}[X_i]=\mu\) and \(\text{V}[X_i] = \sigma^2\). Then as
\(n \rightarrow \infty\), random variable \(\sqrt{n}(\bar{X}_n - \mu)\)
converges in distribution with a normal distribution with mean \(0\) and
variance \(\sigma^2\)

\[\sqrt{n}(\bar{X}_n - \mu) \xrightarrow d N(0,\sigma^2)\]

\hypertarget{more-about-invertible-matrix}{%
\section{More about invertible
matrix}\label{more-about-invertible-matrix}}

\hypertarget{given-suppose-ain-rnxn-and-a-1-exist-then-the-following-can-be-said}{%
\subsection{\texorpdfstring{Given: Suppose \(A\in R^{nxn}\) and
\(A^{-1}\) exist, then the following can be
said}{Given: Suppose A\textbackslash in R\^{}\{nxn\} and A\^{}\{-1\} exist, then the following can be said}}\label{given-suppose-ain-rnxn-and-a-1-exist-then-the-following-can-be-said}}

\begin{itemize}
\tightlist
\item
  The columns of \(A\) is the basis of \(R^{n}\)
\item
  rank \(A\) = n
\item
  \(Nul A=\) \{\(\vec{0}\)\}
\item
  dim \(Nul A\) = 0
\item
  \(A^{-1}A=I\)
\item
  \(AA^{-1}=I\)
\item
  The Linear transformation \(\vec{x} \mapsto A\vec{x}\) is one-to-one
\item
  \(A^T\) is an invertible matrix
\end{itemize}

\hypertarget{change-of-basis}{%
\subsection{Change of basis}\label{change-of-basis}}

Given: \(\vec{y} \notin C(A)\) , and Rank of \(A\) = 2, and
\(\vec{y} \in R^3\)

\hypertarget{problem-1}{%
\subsubsection{Problem 1}\label{problem-1}}

\begin{itemize}
\tightlist
\item
  Let \(\hat{\vec{y}} \subset C(A)\) where
  \(\vec{C}_1 \text{ and } \vec{C}_2\) are the basis of \(C(A)\)
\item
  Find \(\hat{\vec{y}}\) that minimizes \(||\vec{y}-\hat{\vec{y}}||\)
\end{itemize}

\hypertarget{solution}{%
\subsubsection{Solution:}\label{solution}}

\begin{itemize}
\tightlist
\item
  let \(C\) and \(N\) be the matrix that contains the basis of \(C(A)\)
  and \(N(A^T)\)
\item
  Since:
  \(C\vec{x}=\hat{\vec{y}} \text{   and   } C\vec{x} + N\vec{z} = \vec{y}\)
\item
  Simplify the expression
\end{itemize}

\[\begin{aligned}
C^TC\vec{x} &= C^T\vec{y} \\
\vec{x} &= (C^TC)^{-1}C^T\vec{y} 
\end{aligned}\]

\begin{itemize}
\tightlist
\item
  Then,
\end{itemize}

\[C(C^TC)^{-1}C^T\vec{y}=\hat{\vec{y}}\]

\begin{itemize}
\tightlist
\item
  \(C(C^TC)^{-1}C^T\) is called \textbf{projection matrix}*
\end{itemize}

\hypertarget{about-projection-matrix}{%
\section{About projection matrix}\label{about-projection-matrix}}

\[\begin{aligned}
 \mathbb{I}  &= \mathbb{P} + \mathbb{B} \\
\vec{y} &= \mathbb{P}\vec{y} + \mathbb{B}\vec{y}\\
\end{aligned}\]

\begin{itemize}
\tightlist
\item
  where \(P \text{ and } B\) are the \texttt{projection\ matrices} for
  \(C(A) \text{ and } N(A^T)\)
\end{itemize}

\hypertarget{dot-product}{%
\subsection{DOT Product}\label{dot-product}}

\[\hat{\vec{y}} = P_{\vec{u}}^{\vec{y}} = \frac{\vec{y}\centerdot\vec{u}}{\vec{u}\centerdot\vec{u}}\vec{u}\]

where

\(\vec{y}\centerdot\vec{u} \text{ and } \vec{u}\centerdot\vec{u}\) are
scalar quantity.

Projection tells you the \texttt{length} of the
\texttt{projected\ vector}, \(\hat{\vec{y}}\) in terms of the vector
that is \texttt{being\ projected\ onto} \(\vec{u}\)

\begin{Shaded}
\begin{Highlighting}[]
\CommentTok{\# y will be projected onto u}
\NormalTok{y }\OtherTok{\textless{}{-}} \FunctionTok{matrix}\NormalTok{(}\FunctionTok{c}\NormalTok{(}\DecValTok{7}\NormalTok{,}\DecValTok{6}\NormalTok{),}\AttributeTok{nrow=}\DecValTok{2}\NormalTok{)}
\NormalTok{u }\OtherTok{\textless{}{-}} \FunctionTok{matrix}\NormalTok{(}\FunctionTok{c}\NormalTok{(}\DecValTok{4}\NormalTok{,}\DecValTok{2}\NormalTok{),}\AttributeTok{nrow=}\DecValTok{2}\NormalTok{)}
\NormalTok{u0 }\OtherTok{\textless{}{-}} \FunctionTok{matrix}\NormalTok{(}\FunctionTok{c}\NormalTok{(}\DecValTok{16}\NormalTok{,}\DecValTok{8}\NormalTok{),}\AttributeTok{nrow=}\DecValTok{2}\NormalTok{)}
\end{Highlighting}
\end{Shaded}

\hypertarget{example-1}{%
\section{Example}\label{example-1}}

\begin{Shaded}
\begin{Highlighting}[]
\NormalTok{b }\OtherTok{\textless{}{-}} \FunctionTok{c}\NormalTok{(}\DecValTok{1}\NormalTok{,}\DecValTok{5}\NormalTok{,}\DecValTok{3}\NormalTok{)}

\NormalTok{c1 }\OtherTok{\textless{}{-}} \FunctionTok{c}\NormalTok{(}\DecValTok{1}\NormalTok{,}\DecValTok{2}\NormalTok{,}\DecValTok{3}\NormalTok{)}
\NormalTok{c2 }\OtherTok{\textless{}{-}} \FunctionTok{c}\NormalTok{(}\DecValTok{1}\NormalTok{,}\FloatTok{2.1}\NormalTok{,}\FloatTok{3.2}\NormalTok{)}
\NormalTok{c3 }\OtherTok{\textless{}{-}} \FunctionTok{c}\NormalTok{(}\DecValTok{0}\NormalTok{,}\DecValTok{1}\NormalTok{,}\SpecialCharTok{{-}}\DecValTok{2}\NormalTok{)}

\NormalTok{A }\OtherTok{\textless{}{-}} \FunctionTok{cbind}\NormalTok{(c1,c2,c3)}
\FunctionTok{print}\NormalTok{(A)}
\end{Highlighting}
\end{Shaded}

\begin{verbatim}
##      c1  c2 c3
## [1,]  1 1.0  0
## [2,]  2 2.1  1
## [3,]  3 3.2 -2
\end{verbatim}

\begin{Shaded}
\begin{Highlighting}[]
\FunctionTok{Rank}\NormalTok{(A)}
\end{Highlighting}
\end{Shaded}

\begin{verbatim}
## [1] 3
\end{verbatim}

\begin{Shaded}
\begin{Highlighting}[]
\CommentTok{\#get x coordinate}
\NormalTok{x }\OtherTok{\textless{}{-}} \FunctionTok{inv}\NormalTok{(A)}\SpecialCharTok{\%*\%}\NormalTok{b}
\NormalTok{A}\SpecialCharTok{\%*\%}\NormalTok{x}
\end{Highlighting}
\end{Shaded}

\begin{verbatim}
##      [,1]
## [1,]    1
## [2,]    5
## [3,]    3
\end{verbatim}

\hypertarget{example-of-projection-matrix}{%
\subsection{example of projection
matrix}\label{example-of-projection-matrix}}

\begin{Shaded}
\begin{Highlighting}[]
\NormalTok{b }\OtherTok{\textless{}{-}} \FunctionTok{c}\NormalTok{(}\DecValTok{1}\NormalTok{,}\DecValTok{5}\NormalTok{,}\DecValTok{3}\NormalTok{)}

\NormalTok{c1 }\OtherTok{\textless{}{-}} \FunctionTok{c}\NormalTok{(}\DecValTok{1}\NormalTok{,}\DecValTok{2}\NormalTok{,}\DecValTok{3}\NormalTok{)}
\NormalTok{c2 }\OtherTok{\textless{}{-}} \FunctionTok{c}\NormalTok{(}\DecValTok{1}\NormalTok{,}\FloatTok{2.1}\NormalTok{,}\FloatTok{3.2}\NormalTok{)}

\NormalTok{A }\OtherTok{\textless{}{-}} \FunctionTok{cbind}\NormalTok{(c1,c2)}
\FunctionTok{print}\NormalTok{(A)}
\end{Highlighting}
\end{Shaded}

\begin{verbatim}
##      c1  c2
## [1,]  1 1.0
## [2,]  2 2.1
## [3,]  3 3.2
\end{verbatim}

\begin{Shaded}
\begin{Highlighting}[]
\NormalTok{Ab }\OtherTok{\textless{}{-}} \FunctionTok{cbind}\NormalTok{(A,b)}
\FunctionTok{Rank}\NormalTok{(A)}
\end{Highlighting}
\end{Shaded}

\begin{verbatim}
## [1] 2
\end{verbatim}

\begin{Shaded}
\begin{Highlighting}[]
\FunctionTok{Rank}\NormalTok{(Ab)}
\end{Highlighting}
\end{Shaded}

\begin{verbatim}
## [1] 3
\end{verbatim}

\begin{Shaded}
\begin{Highlighting}[]
\CommentTok{\#get x coordinate}
\NormalTok{G }\OtherTok{\textless{}{-}} \FunctionTok{t}\NormalTok{(A)}\SpecialCharTok{\%*\%}\NormalTok{A}

\NormalTok{x }\OtherTok{\textless{}{-}} \FunctionTok{inv}\NormalTok{(G)}\SpecialCharTok{\%*\%}\FunctionTok{t}\NormalTok{(A)}\SpecialCharTok{\%*\%}\NormalTok{b}

\NormalTok{b\_hat }\OtherTok{\textless{}{-}}\NormalTok{ A}\SpecialCharTok{\%*\%}\NormalTok{x}

\CommentTok{\#projection matrix to C(A)}
\NormalTok{P }\OtherTok{\textless{}{-}}\NormalTok{ A}\SpecialCharTok{\%*\%}\FunctionTok{inv}\NormalTok{(G)}\SpecialCharTok{\%*\%}\FunctionTok{t}\NormalTok{(A)}

\NormalTok{b\_hat }\OtherTok{\textless{}{-}}\NormalTok{ P}\SpecialCharTok{\%*\%}\NormalTok{b}

\NormalTok{residual }\OtherTok{\textless{}{-}}\NormalTok{ b}\SpecialCharTok{{-}}\NormalTok{ b\_hat}
\FunctionTok{print}\NormalTok{(residual)}
\end{Highlighting}
\end{Shaded}

\begin{verbatim}
##      [,1]
## [1,]   -1
## [2,]    2
## [3,]   -1
\end{verbatim}

\begin{Shaded}
\begin{Highlighting}[]
\CommentTok{\#projection matrix to left nullspace}
\NormalTok{N}\OtherTok{\textless{}{-}} \FunctionTok{diag}\NormalTok{(}\DecValTok{3}\NormalTok{) }\SpecialCharTok{{-}}\NormalTok{ P}
\NormalTok{N}\SpecialCharTok{\%*\%}\NormalTok{b}
\end{Highlighting}
\end{Shaded}

\begin{verbatim}
##      [,1]
## [1,]   -1
## [2,]    2
## [3,]   -1
\end{verbatim}

\break

\hypertarget{orthogonal}{%
\section{Orthogonal}\label{orthogonal}}

\begin{itemize}
\item
  Two vectors \(\vec{v_1}\) and \(\vec{v_2} \in R^m\) are orthogonal, if
  \(\vec{v_1} \centerdot \vec{v_2}=0\)
\item
  Note that the dot product produce scalar quantity 0 not \(\vec{0}\)
\item
  Notice \(\vec{v}_1\) is size of 3 vector and \texttt{orth(\ )} returns
  normalized \(\vec{v}_1\)
\end{itemize}

\begin{Shaded}
\begin{Highlighting}[]
\NormalTok{v1 }\OtherTok{\textless{}{-}} \FunctionTok{c}\NormalTok{(}\DecValTok{3}\NormalTok{,}\DecValTok{4}\NormalTok{,}\DecValTok{5}\NormalTok{)}
\end{Highlighting}
\end{Shaded}

\hypertarget{normalizing-the-basis}{%
\subsection{Normalizing the basis}\label{normalizing-the-basis}}

\begin{Shaded}
\begin{Highlighting}[]
\NormalTok{c\_A }\OtherTok{\textless{}{-}} \FunctionTok{orth}\NormalTok{(v1)}
\FunctionTok{print}\NormalTok{(c\_A)}
\end{Highlighting}
\end{Shaded}

\begin{verbatim}
##           [,1]
## [1,] 0.4242641
## [2,] 0.5656854
## [3,] 0.7071068
\end{verbatim}

\begin{Shaded}
\begin{Highlighting}[]
\CommentTok{\#notice what happens when you dot v1 and c\_A}
\FunctionTok{print}\NormalTok{(v1}\SpecialCharTok{\%*\%}\NormalTok{c\_A)}
\end{Highlighting}
\end{Shaded}

\begin{verbatim}
##          [,1]
## [1,] 7.071068
\end{verbatim}

\begin{Shaded}
\begin{Highlighting}[]
\FunctionTok{Norm}\NormalTok{(v1)}
\end{Highlighting}
\end{Shaded}

\begin{verbatim}
## [1] 7.071068
\end{verbatim}

\hypertarget{space-subspace-orthogonal-complement-subspace}{%
\subsection{Space, subspace, orthogonal complement
subspace}\label{space-subspace-orthogonal-complement-subspace}}

\begin{itemize}
\tightlist
\item
  Let \(S\) be space of \(R^n\), \(A\) is \(R^{mxn}\) matrix.
\item
  Let \(C(A)\) and \(N(A^T)\) be the column space and left nullspace of
  \(A\)
\item
  \(C(A)\) and \(N(A^T)\) are orthogonal complement subspace of each
  other.
\item
  Then, any vector, \(\vec{x} \in S\) but
  \(\vec{x} \notin C(A) \text{ or } \vec{x} \notin N(A^T)\) can be
  expressed by the linear combination of basis of
  \(C(A) \text{ and } N(A^T)\)
\end{itemize}

\hypertarget{diagonal-matrix}{%
\subsubsection{Diagonal matrix}\label{diagonal-matrix}}

\begin{Shaded}
\begin{Highlighting}[]
\NormalTok{D1 }\OtherTok{\textless{}{-}} \FunctionTok{diag}\NormalTok{(}\FunctionTok{c}\NormalTok{(}\DecValTok{5}\NormalTok{,}\DecValTok{2}\NormalTok{,}\DecValTok{10}\NormalTok{),}\DecValTok{3}\NormalTok{,}\DecValTok{3}\NormalTok{)}
\FunctionTok{print}\NormalTok{(D1)}
\end{Highlighting}
\end{Shaded}

\begin{verbatim}
##      [,1] [,2] [,3]
## [1,]    5    0    0
## [2,]    0    2    0
## [3,]    0    0   10
\end{verbatim}

\begin{Shaded}
\begin{Highlighting}[]
\FunctionTok{print}\NormalTok{(}\FunctionTok{inv}\NormalTok{(D1)) }\CommentTok{\#notice when the diagonal elements has zero in it, D1 becomes singular.}
\end{Highlighting}
\end{Shaded}

\begin{verbatim}
##      [,1] [,2] [,3]
## [1,]  0.2  0.0  0.0
## [2,]  0.0  0.5  0.0
## [3,]  0.0  0.0  0.1
\end{verbatim}

\begin{Shaded}
\begin{Highlighting}[]
\FunctionTok{print}\NormalTok{(D1 }\SpecialCharTok{\%\^{}\%} \DecValTok{3}\NormalTok{) }\CommentTok{\# using the function in expm}
\end{Highlighting}
\end{Shaded}

\begin{verbatim}
##      [,1] [,2] [,3]
## [1,]  125    0    0
## [2,]    0    8    0
## [3,]    0    0 1000
\end{verbatim}

\hypertarget{orthogonal-matrix}{%
\section{Orthogonal matrix}\label{orthogonal-matrix}}

\[U^{-1} = U^T\]

\begin{itemize}
\tightlist
\item
  Let \(W\) be a subspace of \(R^n\) and let \(\vec{y} \in R^n\) but
  \(\vec{y} \notin W\).\\
\item
  Then, \(\hat{\vec{y}} \in W\) that is the closest approximation of
  \(\vec{y}\) is the \(\vec{y}\) projected onto \(W\)
\end{itemize}

\hypertarget{proerty-of-matrx-that-is-not-square-but-has-orthonormal-basis}{%
\subsection{Proerty of matrx that is not square, but has orthonormal
basis}\label{proerty-of-matrx-that-is-not-square-but-has-orthonormal-basis}}

\begin{Shaded}
\begin{Highlighting}[]
\NormalTok{v }\OtherTok{\textless{}{-}} \FunctionTok{matrix}\NormalTok{(}\FunctionTok{c}\NormalTok{(}\DecValTok{2}\NormalTok{,}\DecValTok{1}\NormalTok{,}\DecValTok{2}\NormalTok{),}\AttributeTok{nrow=}\DecValTok{3}\NormalTok{)}
\NormalTok{O }\OtherTok{\textless{}{-}} \FunctionTok{orthonormalization}\NormalTok{(v)}
\FunctionTok{print}\NormalTok{(O)}
\end{Highlighting}
\end{Shaded}

\begin{verbatim}
##           [,1]       [,2]       [,3]
## [1,] 0.6666667 -0.2357023 -0.7071068
## [2,] 0.3333333  0.9428090  0.0000000
## [3,] 0.6666667 -0.2357023  0.7071068
\end{verbatim}

\begin{Shaded}
\begin{Highlighting}[]
\NormalTok{U }\OtherTok{\textless{}{-}} \FunctionTok{cbind}\NormalTok{(O[,}\DecValTok{1}\NormalTok{],O[,}\DecValTok{2}\NormalTok{])}
\FunctionTok{print}\NormalTok{(}\FunctionTok{t}\NormalTok{(U)}\SpecialCharTok{\%*\%}\NormalTok{U)}
\end{Highlighting}
\end{Shaded}

\begin{verbatim}
##      [,1] [,2]
## [1,]    1    0
## [2,]    0    1
\end{verbatim}

Suppose \(C\) is matrix that contains orthonormal basis of \(W\). Since
there exist \(\vec{y} \notin W\), \(C\) can't be square matrix.

However, the basis in \(C\) can still be \texttt{orthonormal}.

Let \(C\) be retangular matrix with orthonormal basis,

\[\vec{y} = C\vec{x}_w + N\vec{x}_N\] where - \(N\) is the basis
spanning orthogonal complement subspace of \(W\). Then,
\[C^{T}\vec{y} = C^TC\vec{x}_w\]

Since \(C\) is matrix that contains orthonormal basis, \(C^TC\) becomes
identify matrix. \[C^{T}\vec{y} = \vec{x}_W \] Now, the location of
\(\hat{\vec{y}}\) in terms of the basis in \(C\) can be expressed as
below \[C\vec{x}_W = \hat{\vec{y}}\]

Solving for \(\vec{x}_W\) \[\vec{x}_W = C^T\hat{\vec{y}}\]

Sub the above expression of \(\vec{x}_W\) to the following equation

\[C^{T}\vec{y} = C^TC\vec{x}_w\]

\[C^{T}\vec{y} = C^TC(C^T\hat{\vec{y}})\]

Then,

\[CC^T\vec{y} = \hat{\vec{y}}\]

\hypertarget{gram-schmidt-process}{%
\section{Gram-Schmidt Process}\label{gram-schmidt-process}}

\begin{itemize}
\item
  Let \{\(\vec{x}_1,\vec{x}_2...\vec{x}_p\)\} be basis for a nonzero
  subspace \(W\) of \(R^n\) where \(p < n\). Gram-Schimidt process
  converts \{\(\vec{x}_1,\vec{x}_2...\vec{x}_p\)\} to
  \{\(\vec{v}_1,\vec{v}_2...\vec{v}_p\)\} where
  \{\(\vec{v}_1,\vec{v}_2...\vec{v}_p\)\} are orthogonal basis for \(W\)
\item
  Gram-Schimit process is projecting one set of basis to another basis
  that is orthogonal to them.
\item
  Notice the \texttt{orthonormalization(\ )} in \texttt{R} returns 3 x 3
  matrix. This function in R returns the basis spanning the subspace
  that is orthogonal to subspace spanned by \(\vec{v}_1\)
\end{itemize}

\begin{Shaded}
\begin{Highlighting}[]
\NormalTok{GS }\OtherTok{\textless{}{-}} \FunctionTok{orthonormalization}\NormalTok{(v1)}
\FunctionTok{print}\NormalTok{(GS)}
\end{Highlighting}
\end{Shaded}

\begin{verbatim}
##           [,1]       [,2]       [,3]
## [1,] 0.4242641  0.9055385  0.0000000
## [2,] 0.5656854 -0.2650357  0.7808688
## [3,] 0.7071068 -0.3312946 -0.6246950
\end{verbatim}

\href{https://cran.r-project.org/web/packages/matlib/vignettes/gramreg.html}{Gram-Schmidt}

\end{document}
